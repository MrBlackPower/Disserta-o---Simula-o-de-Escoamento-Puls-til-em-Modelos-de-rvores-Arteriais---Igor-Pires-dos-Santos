\chapter{RESULTADOS NUMÉRICOS}\label{sec:resultados}

%\begin{itemize}
%	\item \textcolor{red}{IGOR: já fiz uma primeira revisão deste capítulo e indico em vermelho o que deve fazer para melhorar este capítulo. Tudo que acrescentar ou mudar coloque em azul.} \textcolor{green}{OK}
%	\item \textcolor{red}{IGOR: não está aparecendo a numeração das Figuras no caption. Algo de errado está acontecendo no modelo que está utilizando.} \textcolor{green}{OK}
%	\item \textcolor{red}{IGOR: está errada a citação do trabalho do método de Duan e Zamir (1995), não é a citação que colocou \cite{Duan}.}\textcolor{green}{\textbf{OK}}
%\end{itemize}

Nesta seção, apresentam-se resultados obtidos com a implementação computacional e simulação do modelo matemático de Duan e Zamir~\cite{Duan1992}. As simulações realizadas aqui tratam da propagação de uma onda harmônica simples ao longo de uma árvore, onde reflexões de onda modificam a amplitude da onda de pressão enquanto ela avança. A escolha de uma onda harmônica simples neste estudo possibilita investigar os efeitos da frequência, fluido viscoso e viscoelasticidade da parede do segmento de vaso.

Considerou-se neste estudo um modelo de árvore arterial canina como ilustrado na Figura~\ref{fig:arvore-canina}. As propriedades dos segmentos foram escolhidas oriundas dos dados de Fung~\cite{fung2013biomechanics} e são descritas na Tabela~\ref{tab1:proprerty}. 

\begin{figure}[!htbp]
	\centering
	\includegraphics[scale=0.8]{Figures/tree_canine.png}
	\caption{Representação do modelo de árvore arterial canina (figura adaptada de~\cite{Duan}).}
	\label{fig:arvore-canina}
\end{figure}

\begin{table}[!htbp]
	\caption{Propriedades dos segmentos do modelo de árvore arterial~\cite{Duan,Fung}}
	\centering{}
	\begin{tabular}{c|c|c|c|c|c}
		\toprule 
		Artéria	& Comprimento & Densidade & Viscosidade  & Diâmetro & Módulo de  \\ 
		& ($cm$) & $\rho$ ($g/cm^3$) & $\mu_0$ ($g/cm s$) & ($cm$) & Young ($dyn/cm^2$) \\ 
		\midrule 
		Aorta & 25 & $0,960$ & 0,0385 & 1,3 &4,8 $\times 10^6$ \\ 
		Descendente &  & &  & & \\ 
		\hline 
		Aorta & 11 & $1,134$ & 0,0449 & 0,9 & 1,0 $\times 10^7$ \\
		Abdominal &  & &  & &  \\ 
		\hline 
		Ilíaca & 12 & $1,172$ & 0,0472 & 0,6 & 1,0 $\times 10^7$\\ 
		\hline 
		Femoral & 10 & $1,235$ & 0,0494 & 0,4 & 1,0 $\times 10^7$\\ 
		\bottomrule 
	\end{tabular} 
	\label{tab1:proprerty}
\end{table}

Nas simulações realizadas, calculou-se a distribuição de amplitude de pressão ao longo da árvore arterial (Figura~\ref{fig:arvore-canina}). Os resultados foram obtidos para quatro diferentes frequên\-cias e três diferentes cenários de escoamento/segmento: (i) escoamento viscoso em segmento puramente elástico (cenário 1 da Seção~\ref{sec:cenario}), (ii) escoamento invíscido em segmento viscoelástico (cenário 2) e (iii) escoamento viscoso em segmento viscoelástico (cenário 3). 

Os resultados obtidos nas simulações são mostrados nas Figuras~\ref{fig3a:arterial-tree}, \ref{fig3b:arterial-tree}, \ref{fig4a:arterial-tree}, \ref{fig4b:arterial-tree}, \ref{fig5a:arterial-tree} envolvendo a amplitude da pressão ao longo do modelo de árvore arterial. Nestas figuras, o comprimento de cada segmento arterial foi dimensionado para $1,0$, de modo que o comprimento adimensional total da árvore é $4,0$. O comprimento real é $58$ cm. A amplitude da pressão também foi escalada pela pressão de entrada $P_o$, e os resultados finais são portanto mostrados em termos de amplitude de pressão adimensional $|P|$ versus a distância adimensional $X$ do início da árvore.


%--------------------------------------------------------------------------------%
\section{ESCOAMENTO VISCOSO}\label{sec:cenario1}

Nas Figuras~\ref{fig3a:arterial-tree} e \ref{fig3b:arterial-tree}, o efeito da viscosidade do fluido é examinado separadamente conside\-ran\-do-se o escoamento em segmentos puramente elásticos com quatro valores diferentes de viscosidade do fluido, ou seja, $\mu = 0$; $0,5 \mu_0$; $1,0 \mu_0$ e $1,5 \mu_0$, onde $\mu_0$ é o valor base da viscosidade da Tabela~\ref{tab1:proprerty}. Observa-se que o efeito da viscosidade do fluido é reduzir o aumento global na amplitude da onda de pressão causada pelas reflexões das ondas à medida que a onda se desloca na direção à jusante. Além disso, modera os picos locais na distribuição de pressão.

\begin{figure}[!htbp]
	\centering
	(a) \\
	\includegraphics[scale=0.7]{Figures/fig3_P_f3_65_visc_NEW.png}\\
	(b)\\
	\includegraphics[scale=0.7]{Figures/fig3_P_f7_30_visc_NEW.png}\\
	\caption{Amplitude da pressão $|P|$ ao longo da árvore arterial considerando diferentes viscosidade do fluido $\mu$ e frequências: (a) $f$ = 3,65 Hz, (b)  $f$ = 7,30 Hz. }
	\label{fig3a:arterial-tree}%
\end{figure}

\begin{figure}[!htbp]
	\centering
	(a) $f$ = 10,95 Hz\\
	\includegraphics[scale=0.7]{Figures/fig3_P_f10_95_visc_NEW.png}\\
	(b) $f$ = 14,60 Hz\\
	\includegraphics[scale=0.7]{Figures/fig3_P_f14_60_visc_NEW.png}\\
	\caption{Amplitude da pressão $|P|$ ao longo da árvore arterial considerando diferentes viscosidade do fluido $\mu$ e frequências: (a) $f$ = 10,95 Hz, (b)  $f$ = 14,60 Hz. }
	\label{fig3b:arterial-tree}%
\end{figure}

%--------------------------------------------------------------------------------%
\section{SEGMENTO VISCOELÁSTICO}\label{sec:cenario2}

Nas Figuras~\ref{fig4a:arterial-tree}, \ref{fig4b:arterial-tree}, o efeito da viscoelasticidade da parede do segmento de vaso é considerado separadamente considerando-se o escoamento invíscido e tomando-se quatro valores diferentes da viscoelasticidade da parede do segmento. O modelo viscoelástico proposto utilizado para fins destes cálculos é apresentado no cenário 2 da Seção~\ref{sec:cenario}, no qual a viscoelasticidade da parede do vaso é representada por um módulo de Young complexo. Estas figuras mostram os resultados para $\phi_0$ = $0^o$, $4^o$, $8^o$ e $12^o$. Quando $\phi_0$ = $0^o$ tem-se um valor representando uma parede puramente elástica e para $\phi_0> 0$ tem-se a representação da viscoelasticidade. Nota-se a partir destas figuras que o efeito da viscoelasticidade, como o da viscosidade do fluido, é amortecer o aumento global da amplitude da onda de pressão causada pelas reflexões das ondas à medida que a onda se desloca na direção à jusante, bem como moderar os picos locais na distribuição de pressão. 

\begin{figure}[!htbp]
	\centering
	(a) $f$ = 3,65 Hz \\
	\includegraphics[scale=0.7]{Figures/fig4_P_f3_65_viscoelasticity_NEW.png}\\
	(b)  $f$ = 7,30 Hz\\
	\includegraphics[scale=0.7]{Figures/fig4_P_f7_30_viscoelasticity_NEW.png}\\
	\caption{Amplitude da pressão $|P|$ ao longo da árvore arterial considerando diferentes valores de viscoelasticidade $\phi_0$ e frequências: $f$ = 3,65 Hz e  $f$ = 7,30 Hz. }
	\label{fig4a:arterial-tree}%
\end{figure}

\begin{figure}[!htbp]
	\centering
	(a) $f$ = 10,95 Hz\\
	\includegraphics[scale=0.7]{Figures/fig4_P_f10_95_viscoelasticity_NEW.png}\\
	(b) $f$ = 14,60 Hz\\
	\includegraphics[scale=0.7]{Figures/fig4_P_f14_60_viscoelasticity_NEW.png}\\
	\caption{Amplitude da pressão $|P|$ ao longo da árvore arterial considerando diferentes valores de viscoelasticidade $\phi_0$ e frequências: $f$ = 10,95 Hz e $f$ = 14,60 Hz.}
	\label{fig4b:arterial-tree}%
\end{figure}

%--------------------------------------------------------------------------------%
\section{ESCOAMENTO VISCOSO EM SEGMENTO VISCOELÁSTICO}\label{sec:cenario3}

Nas Figuras~\ref{fig5a:arterial-tree}, \ref{fig5b:arterial-tree}, o efeito da viscoelasticidade da parede do segmento é adicionada ao efeito do escoamento viscoso, adotando dois valores diferentes da viscoelasticidade e dois valores de viscosidade. O modelo utilizado no cenário 3 é a soma dos efeitos apresentados na Seção~\ref{sec:cenario}, adicionando o fator viscoso e o módulo de Young complexo. Estas figuras mostram o resultado para $\phi_0$ = $0^o$, $8^o$ e com as viscosidades $\mu = 0$ e $0,5 \mu_0$. Ao visualizar os efeitos da viscoelasticidade e viscosidade na amplitude da onde de Pressão $P$, é observado o amortecimento global da amplitude de onda de pressão, somado à moderação dos picos locais na distribuição de pressão. Entretanto, o efeito somado dos fenômenos causa um amortecimento mais eficaz que o visto nos outros cenários.

\begin{figure}[!htbp]
	\centering
	(a) \\
	\includegraphics[scale=0.7]{Figures/fig5_P_f3_65_viscoelasticity_viscosity_NEW.png}\\
	(b)\\
	\includegraphics[scale=0.7]{Figures/fig5_P_f7_30_viscoelasticity_viscosity_NEW.png}\\
	\caption{Amplitude da pressão $|P|$ ao longo da árvore arterial considerando diferentes valores de viscoelasticidade $\phi_0$ e frequências: (a) $f$ = 3,65 Hz, (b)  $f$ = 7,30 Hz. }
	\label{fig5a:arterial-tree}%
\end{figure}

\begin{figure}[!htbp]
	\centering
	(a) $f$ = 10,95 Hz\\
	\includegraphics[scale=0.7]{Figures/fig5_P_f10_95_viscoelasticity_viscosity_NEW.png}\\
	(b) $f$ = 14,60 Hz\\
	\includegraphics[scale=0.7]{Figures/fig5_P_f14_60_viscoelasticity_viscosity.png}\\
	\caption{Amplitude da pressão $|P|$ ao longo da árvore arterial considerando diferentes valores de viscoelasticidade $\phi_0$ e frequências: (a) $f$ = 10,95 Hz, (b) $f$ = 14,60 Hz.}
	\label{fig5b:arterial-tree}%
\end{figure}

%--------------------------------------------------------------------------------%
\chapter{RESULTADOS COMPUTACIONAIS}\label{sec:resultados2}

Nesta seção, apresentam-se resultados obtidos com a implementação da ferramenta computacional em seus dois ambientes \textit{IGU} e \textit{InGU}. As simulações realizadas aqui tratam da aplicação da ferramenta na obtenção de resultados do escoamento pulsátil em um modelo de árvore arterial. Os resultados de ambas as versões serão apresentados e comparados com a versão anteriormente apresentada.

O arquivo \textit{CMakeLists.txt} é o responsável por interpretar o projeto \textit{Qt} e corretamente compilar as bibliotecas necessárias. Na interface \textit{IDE} do \textit{QtCreator} este arquivo contém as instruções para compilar todos os ambientes deste trabalho e um ambiente com a versão anterior da ferramenta computacional. A versão anterior da ferramenta computacional \textit{IGU} exporta seus resultados em arquivos \textit{VTK}, estes arquivos podem ser lidos pela nova versão da ferramenta ou ainda utilizados em outro ambiente.


\begin{figure}[!htbp]
	\centering
	\includegraphics[scale=1.5]{Figures/cmake_print.png}
	\caption{Representação do projeto aberto através do arquivo \textit{CMakeLists.txt} na interface \textit{IDE} do \textit{QtCreator}.}
	\label{fig:cmake}
\end{figure}


Ao abrir o arquivo do projeto é possível verificar a existência de três projetos à serem compilados: \textit{IGU}, contendo a versão anterior antiga do progama. \textit{IGU2}, que contém a versão da ferramenta computacional apresentada neste trabalho com interface gráfica; E \textit{InGU}, contendo a estrutura de dados mais recente sem uma interface gráfica. Podemos ver estes projetos presentes na Figura~\ref{fig:cmake}.

Ao executar a versão mais antiga da ferramenta computacional uma estrutura de dados diferente é utilizada no processo de compilação. Isto porque os ambientes mais recentes disponibilizados \textit{IGU} e \textit{InGU} utilizam as mesmas classes e a mesma estrutura de dados.


\begin{figure}[!htbp]
	\centering
	\includegraphics[width=\linewidth]{Figures/IGU_old2.png}
	\caption{Ferramenta computacional \textit{IGU} em sua versão mais antiga.}
	\label{fig:IGU-old}
\end{figure}

Na Figura~\ref{fig:IGU-old} observamos como a interface gráfica do usuário possui duas telas de \textit{Canvas} e diversos botões na janela principal. O funcionamento desta interface é muito similar ao da nova interface, entretanto está muito mais ligada exclusivamente à simulação do escoamento pulsátil. Dois \textit{Canvas} possibilitam a exibição de uma árvore arterial e seu gráfico ao mesmo tempo, entretanto prende estes dois elementos na janela principal. Com o esquema de janelas desenvolvido na nova versão da ferramenta é possível que diversos objetos gráficos sejam exibidos ao mesmo tempo e estas janelas escaladas independentemente.

A estrutura de dados presente na versão anterior da ferramenta computacional não apresenta os conceitos de fábrica e é incapaz de armazenar todos os estados do modelo geométrico, caso deseje-se recuperar um estado anterior é necessário reiniciar o experimento.

Outra grande diferença é que o processo iterativo dos modelos estava atrelada ao uso da interface de usuário, sem a possibilidade de executar uma sequência de comandos de uma só vez. Portanto não é possível utilizar a ferramenta computacional nesta versão precisando seu tempo de execução porque o seu processo de configuração requer interação com a interface de usuário. Através dos comandos disponibilizados na Seção~\ref{sec:console}, presentes na versão mais atual da ferramenta, é possível que objetos inteligentes executem uma bateria de comandos e seu desempenho mensurado e analisado.

Finalmente, como visto na Seção~\ref{sec:threads} a ferramenta utiliza a \textit{thread} \textit{WiseProcessor} para processar o objeto inteligente \textit{WiseObject}, este fator permite que ferramenta execute o ciclo de iteração sem afetas o funcionamento da interface gráfica de usuário ou bloquear seus recursos.

Para quantificar a melhoria de desempenho trazida pela arquitetura proposta um experimento foi elaborado. Nas simulações realizadas, calculou-se a distribuição de amplitude de pressão ao longo da árvore arterial(Figura~\ref{fig:arvore-canina}) e resultados variando três parâmetros foram executados: (i) viscosidade $\mu$, (ii) ângulo de faze $\phi$ e (iii) frequência $f$, refazendo os três cenários apresentados na Seção~\ref{sec:resultados}.

\begin{table}[!htbp]
	\caption{Parâmetros de entrada utilizados no teste de carga.}
	\centering{}
	\begin{tabular}{c|c}
		\toprule 
		Artéria	& $f \in \{3.65,7.30,10.95,14.60\}$  \\
		\midrule 
		Viscosidade	& $\mu \in \{0,0.5\mu_0,1.0\mu_0,1.5\mu_0\}$    \\ 
		\midrule 
		Ângulo de Fase	& $\phi_0 \in \{0{\circ},4{\circ},8{\circ},12{\circ}\}$  \\ 
		\bottomrule 
	\end{tabular} 
	\label{tab1:entrada}
\end{table}

Na tabela \ref{tab1:entrada} estão os parâmetros de entrada utilizados para testar o desempenho da ferramenta computacional. O objetivo é realizar $64$ iterações combinando os parâmetros de entrada e armazenar o tempo de execução, cada iteração irá executar o experimento com uma combinação dos três parâmetros de entrada.

O gerenciador de \textit{threads} inteligente \textit{WiseThreadPool} é o responsável por gerenciar a quantidade de \textit{threads} utilizada pela ferramenta computacional. Primeiramente observou-se que o modelo matemático aplicado ao modelo geométrico escolhido possui uma média tempo de execução de $30ms$, enquanto uma operação de escrita e leitura demora entre$100$ e $300ms$. Como o ciclo de iteração de um objeto inteligente \textit{WiseObject} envolve diretamente operações de escrita e leitura, estas foram as \textit{threads} escolhidas à serem duplicadas. \textit{Threads} do tipo \textit{WiseConsole} e \textit{WiseProcessor} requerem uma quantidade maior de trabalhos para que se obtenha algum aumento de desempenho, no experimento escolhido o aumento da quantidade destas \textit{threads} implica diretamente no aumento do tempo de execução devido à comunicação entre as \textit{threads}.

O experimento foi executado em uma máquina com o processador \textit{Intel Core I9-9900K@3.60GHz} com $4GB@2300MHz$ de memória RAM disponíveis. Para medir o tempo de execução o ambiente computacional \textit{InGU}, por não possuir interface gráfica demonstrou ser o mais rápido durante os testes.


\begin{figure}[!htbp]
	\centering
	\includegraphics[width=\linewidth]{Figures/INGU.png}
	\caption{Ferramenta computacional \textit{InGU} executando o comando de leitura de arquivo de entrada.}
	\label{fig:INGU}
\end{figure}

Para executar o teste um arquivo de entrada com uma lista de comandos, assim como o Anexo~\ref{annex2}, foi confeccionado para criar a estrutura, realizar os cálculos, salvar os resultados e limpar o ambiente. O ambiente computacional \textit{InGU} permite sua execução com parâmetros de entrada. Ao executar o comando "\textit{./INGU read cmds arquivo}" o ambiente computacional executa o comando de leitura de arquivo de entrada e finaliza o programa. A Figura~\ref{fig:INGU} mostra este comportamento.

Para que as \textit{threads} aumentem o desempenho, os comando precisam ser divididos em cargas de trabalho. Os comandos recebidos pela ferramenta são executados de forma sequencial dentro de seus respectivos grupos de trabalho. Utilizando objetos inteligente \textit{WiseObject} distintos, grupos de trabalho diferentes podem ser executados simultaneamente. Com isto o experimento foi dividido em $4$ grupos, com o número de grupo de trabalhos $w \in {1,2,4,8}$ ob tendo o tempo de execução para o número de \textit{threads} $t \in {1,2,4,8}$. Como as cargas de trabalho não compartilham objetos inteligentes \textit{WiseObject}, o uso de mais cargas afeta diretamente a quantidade de memória utilizada pela ferramenta computacional.


%--------------------------------------------------------------------------------%
\section{MELHORA DE DESEMPENHO DE THREADS}\label{sec:cenario4}

Na Tabela~\ref{tab1:medium} estão registrados os tempos médios de execução do experimento.

\begin{table}[!htbp]
	\caption{Tempo de execução média em milissegundos $ms$ do ciclo de iteração com diferentes arranjos de \textit{threads}, em negrito os melhores tempos.}
	\centering{}
	\begin{tabular}{c|c|c|c|c}
		\toprule 
		\textbf{Cargas de Trabalho}	& $1$ & $2$ & $4$  & $8$\\ 
		\midrule 
		\textbf{1 Thread} & 7854,4 &	4475,4 &	2810,6 &	2453,6\\ 
		\midrule 
		\textbf{2 Threads} & \textbf{5078,8} &	\textbf{3543,6} &	2885,8 &	2564,4\\ 
		\midrule 
		\textbf{4 Threads} & 5232,4 &	3286 &	\textbf{2239,8} &	\textbf{1959,6}\\ 
		\midrule 
		\textbf{8 Threads} & 7677 &	4545,4 &	2887,8 &	2452,6
		
		\\ 
		\bottomrule 
	\end{tabular} 
	\label{tab1:medium}
\end{table}

Com estes tempos de execução construiu-se Tabela~\ref{tab1:speedup} contendo o cálculo do ganho de performance ao duplicar o número de \textit{threads}. O \textit{speed up} $S_th$ caculado nesta tabela quantifica é dado por:

\begin{equation}
	S_th = t_{th-1}/t_{th},
	\label{eq:speedup1}
\end{equation}
sendo $th \in \{1,2,4,8\}$ a quantidade de \textit{threads}.

\begin{table}[!htbp]
\caption{\textit{Speed up} do ciclo de iteração com diferentes arranjos de \textit{threads}, em negrito os melhores aumentos de desempenho.}
\centering{}
\begin{tabular}{c|c|c|c|c}
	\toprule 
	\textbf{Cargas de Trabalho}	& $1$ & $2$ & $4$  & $8$\\ 
	\midrule 
	\textbf{2 Threads} & \textbf{1,54650} &	\textbf{1,26295} & 0,97394 &	0,95679\\ 
	\midrule 
	\textbf{4 Threads} & 0,970641 &	1,07839 &	\textbf{1,28841} & \textbf{1,30863}\\ 
	\midrule 
	\textbf{8 Threads} & 0,68156 &	0,72292 & 0,77560 & 0,79898
	\\ 
	\bottomrule 
\end{tabular} 
\label{tab1:speedup}
\end{table}

Ao duplicar o número de \textit{threads} é esperado um \textit{speed up} de $2$, entretanto a comunicação entre as \textit{threads} não permite que este número seja alcançado; Após uma quantidade de \textit{threads} este número passa a cair com a adição de novas \textit{threads}. Observamos que no caso da ferramenta computacional, reorganizar o experimento em cargas de trabalho é capaz de aumentar a velocidade da execução até mesmo em sistema com apenas uma \textit{thread}.

Os resultados obtidos com $2$ e $4$ \textit{threads} foram os melhores, com esta quantidade de \textit{threads} e com o aumento da quantidade de cargas de trabalho é possível melhorar drasticamente o desempenho da ferramenta. Com $8$ \textit{threads} a ferramenta computacional apresenta um tempo médio de execução mesmo com uma quantidade maior de trabalhos.

%--------------------------------------------------------------------------------%
\section{MELHORA DE DESEMPENHO DE CARGA DE TRABALHO}\label{sec:cenario5}

Com os mesmos resultados de tempo apresentados na Tabela~\ref{tab1:medium} foi calculado a melhora de desempenho pela utilização de uma quantidade maior de cargas de trabalho $w$. O \textit{speed up} $S_w$ caculado nesta tabela quantifica é dado por:

\begin{equation}
	S_w = t_{w-1}/t_{w},
	\label{eq:speedup2}
\end{equation}
sendo $w \in \{1,2,4,8\}$ a quantidade cargas de trabalho.

\begin{table}[!htbp]
\caption{\textit{Speed up} do ciclo de iteração com diferentes arranjos de cargas de trabalho $w$, em negrito os melhores aumentos de desempenho.}
\centering{}
\begin{tabular}{c|c|c|c}
	\toprule 
	\textbf{Cargas de Trabalho} & $2$ & $4$  & $8$\\ 
	\midrule 
	\textbf{1 Thread} & \textbf{1,755} &	\textbf{1,5923} &	\textbf{1,1455} \\ 
	\midrule 
	\textbf{2 Threads} & 1,4332 &	1,2279 & 1,1253\\ 
	\midrule 
	\textbf{4 Threads} & 1,5923 &	1,4671 &	1,143\\ 
	\midrule 
	\textbf{8 Threads} & 1,689 &	1,574 & 1,1774	\\ 
	\bottomrule 
\end{tabular} 
\label{tab1:speedup2}
\end{table}

Ao duplicar o número de cargas de trabalho $w$ é observada uma melhora expressiva no ambiente executado com uma \textit{thread}, mostrando que a ferramenta é capaz de tirar proveito dos recursos da máquina mesmo que não hajam diversos núcleos de processamento disponíveis. Diferentemente do \textit{speed up} calculado no aumento de \textit{threads}, neste caso há sempre uma melhora de desempenho.