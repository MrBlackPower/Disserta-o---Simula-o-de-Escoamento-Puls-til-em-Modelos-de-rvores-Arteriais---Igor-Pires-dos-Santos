\chapter{INTRODUÇÃO}\label{sec:intro}  %%Nesta linha, dentro de { }, digita-se em CAIXA ALTA, como apresentado aqui

Estudos de simulação hemodinâmica têm sido frequentemente baseados em modelos de árvores arteriais para obter uma melhor compreensão de todos os aspectos relacionados ao escoamento sanguíneo, desde a propagação de ondas e análise do pulso de pressão, passando pelo diagnóstico e inclusive com aplicações no planejamento cirúrgico. Como a representação do sistema cardiovascular através de um modelo puramente 3D que leve em conta a estrutura geométrica exata de todos os vasos não é, no momento, viável computacionalmente, vêm sendo empregados modelos dimensionalmente heterogêneos conhecidos como 0D (zero-dimensional)--1D (unidimensional)--2D (bidimensional)--3D (tridimensional) \cite{Formaggia2001}. 

Modelos 3D \cite{Peskin1972,Taylor1998} são utilizados para estudar em detalhe a hemodinâmica local de distritos arteriais de interesse, e a geometria destes modelos são provenientes de dados anatômicos obtidos normalmente via reconstrução de imagens médicas de pacientes específicos. Modelos 1D \cite{Avolio,Formaggia2003,Stergiopulos1992} são adotados para representar as artérias de maior calibre e a estrutura geométrica destes modelos pode ser construída a partir de dados anatômicos. Tais modelos são capazes de capturar os efeitos de propagação de ondas~\cite{Anliker1971,Duan}, a interação das reflexões destas ondas e dar como resultado um pulso de pressão e vazão com significado fisiológico tanto em artérias centrais como periféricas. No entanto, um modelo 1D de toda a árvore arterial sistêmica não é possível devido à falta de dados anatômicos precisos das regiões periféricas. Portanto, a árvore tem que ser truncada em algum nível. Normalmente, este truncamento é feito empregando modelos 0D \cite{Mates1988,Stergiopulos1992} conhecidos por terminais Windkessel à jusante da posição distal do modelo 1D para representar o comportamento de distritos arteriais relacionados com o nível de arteríolas e capilares. 

Os modelos matemáticos são reconhecidos como boas ferramentas de medição tendo seus resultados comparados com dados in-vivo~\cite{BERTAGLIA2020109595}. Para simular corretamente a hemodinâmica corporal, as propriedades viscoelásticas do vaso precisam ser consideradas no comportamento do escoamento cardiovascular. Apesar disso, além da dificuldade de obter dados anatômicos precisos, a quantidade de cálculos necessários e a complexidade matemática e numérica fazem do desenvolvimento de ferramentas confiáveis e práticas para a simulação do sistema cardiovascular humano um dos desafios das próximas décadas~\cite{QUARTERONI20043}.

%\textcolor{red}{IGOR: os dois parágrafos acima estão legais no sentido da ideia para se colocar em uma introdução, mais necessitam ser um pouco reescritos e principalmente, citar referências mais recentes de estudos hemodinâmicas envolvendo modelos 3D, 1D, 0D, e modelos acoplados 0D-1D, 3D-1D-0D. Vale a pena buscar estas referências e ler a introdução deste artigos}\textcolor{green}{OK}

A incidência maior de picos na onda de pressão ao percorrer a aorta já foi documentada como evidência para os efeitos da reflexão em árvores vasculares \cite{Kouchoukos,Lighthill,McDonald}. Enquanto as áreas de reflexão não podem ser completamente conhecidas ou localizadas, é geralmente aceito que a forma da onda de pressão é modificada significativamente enquanto progride pela aorta, de uma forma que só pode ser explicada por reflexões de onda.  Um entendimento mais claro da relação entre modificações e fatores de modificação motiva a busca e desenvolvimento de modelos matemáticos que determinam a forma da onda que o pulso de pressão toma em cada ponto ao percorrer uma árvore arterial. 

Dentro deste contexto, adotou-se neste trabalho o modelo matemático de Duan e Zamir ~\cite{Duan,Duan1992} que descreve escoamento sanguíneo pulsátil em árvores arteriais. Estes autores propuseram um modelo relativamente simples para representação da pressão sanguínea e do fluxo em um modelo de árvore arterial. Dentro de cada segmento de vaso, o escoamento sanguíneo foi calculado baseado em uma aproximação de Womersley, incluindo a elasticidade da parede, bem como a densidade do sangue e a viscosidade.

A capacidade de capturar o pico de pressão existente no escoamento sanguíneo justifica a escolha do modelo matemático de Duan e Zamir para implementação e simulação computacional. Este modelo possibilita o cálculo correto das características locais das ondas de pressão e fluxo a medida que elas progridem ao longo de um modelo de árvore 1D e se tornam modificadas por reflexões de onda.

A presença de picos de pressão, a velocidade da onda e a pressão da reflexão de onda podem indicar a existência problemas na circulação periférica~\cite{TAKAHASHI202129}. Portanto, através do modelo apresentado é possível analisar os modelos geométricos de árvores arteriais, permitindo sugerir e planejar intervenções subsequentes, sem a necessidade de procedimentos invasivos.

%\textcolor{red}{IGOR: escreva um parágrafo de trabalhos da literatura que citam e utilizam o modelo de Duan e Zamir. Busque referências na literatura. Lembrando que esta referência \cite{Duan} está errada. A referência correta é de 1995. Principalmente, cite o seu trabalho publicado na revista Mundi deste parágrafo.}\textcolor{green}{OK}

%--------------------------------------------------------------------------------%
\section{OBJETIVOS}\label{sec:obj}

Os objetivos que norteiam este trabalho são:
\begin{itemize}
	\item desenvolver uma ferramenta computacional capaz de simular o modelo matemático de Duan e Zamir~\cite{Duan1992}
	\item aplicar a ferramenta desenvolvida considerando diferentes cenários hemodinâmicos para investigar os efeitos da viscosidade sanguínea e da viscoelasticidade da parede do vaso no escoamento sanguíneo.
	\item aplicar conceitos da linguagem orientada à objetos C++ à ferramenta computacional, bem como disponibilizar uma interface gráfica através da adição de bibliotecas complementares OpenGL/Qt.
	\item generalizar a estrutura de dados da ferramenta computacional para que ela possa ser utilizada futuramente, no estudo de árvore arteriais e na visualização dos resultados de uma simulação computacional.
	\item contribuir com a literatura ao reproduzir experimentos de escoamento pulsátil, propor um modelo de classes com a adição de funcionalidades modernas~\cite{factorypattern,QTClasses} e, finalmente, entregar uma ferramenta computacional capaz de analisar modelos geométricos similares.
\end{itemize}

%--------------------------------------------------------------------------------%
\section{ORGANIZAÇÃO}\label{sec:org}

Os demais capítulos deste trabalho estão organizados como segue:
\begin{itemize}
	\item Capítulo 2 -
	Primeiramente, neste capítulo é apresentado o modelo matemático utilizado para calcular os efeitos do escoamento pulsátil ao atravessar uma árvore arterial sobre três cenários diferentes, o primeiro sobre o efeito da viscoelasticidade dos vasos sanguíneos, em seguida da viscosidade sanguínea e, finalmente, um cenário com ambos os efeitos.
	
	\item Capítulo 3 -
	Neste capítulo, apresenta-se a ferramenta computacional desenvolvida em C++, a qual conta com a utilização das bibliotecas Qt/OpenGL acrescentada de funcionalidades da programação paralela, orientação à objetos e utilização de estruturas modernas~\cite{factorypattern}.
	
	\item Capítulo 4 -
	O quarto capítulo discorre sobre os resultados numéricos obtidos e uma breve discussão é apresentada. Os resultados numéricos obtidos pela ferramenta computacional proposta, em cima de sua estruturas de dados própria, foram comparados com os resultados obtidos pela literatura.
	
	\item Anexos -
	A seção de anexos contém informações complementares ao uso da ferramenta. O Anexo~\ref{annex1} contém os passos necessários para se compilar a ferramenta computacional. O Anexo~\ref{annex2} demonstra o formato esperado de um arquivo de comandos à ser utilizado. Os Anexos~\ref{annex3} e \ref{annex4} representam os arquivos de saída da ferramenta computacional, um elemento e um objeto inteligente, respectivamente.
\end{itemize}

Este trabalho num primeiro momento descreverá modelo matemático e o esquema iterativo e os conceitos matemáticos que regem o escoamento sanguíneo. Em seguida, propõe uma ferramenta computacional cujo principal objetivo é resolver e visualizar os efeitos da reflexão de onda no escoamento sanguíneo sobre diferentes condições. As tecnologias inseridas na ferramenta computacional permitem que ela seja altamente configurável em tempo de execução. Ao descrever o funcionamento da ferramenta computacional, o modelo matemático e geométrico é constantemente validado, pois o principal objetivo da ferramenta computacional é reproduzir corretamente o escoamento sanguíneo nestes cenários.

%\textcolor{red}{IGOR: complemente esta seção de organização. Recomendo fortemente ler a introdução da minha dissertação e tese para ter mais ideias de como montar uma introdução para sua dissertação.}\textcolor{green}{OK}