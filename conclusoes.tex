\chapter{CONCLUSÕES E TRABALHOS FUTUROS}\label{sec:conclusoes}
\justifying

%\textcolor{red}{IGOR: já fiz uma primeira revisão deste capítulo e indico em vermelho o que deve fazer para melhorar este capítulo. Tudo que acrescentar ou mudar coloque em azul.} \textcolor{green}{OK}
%\textcolor{blue}{XXXXXXXXXXXXX}.

Este trabalho apresenta uma ferramenta computacional construída para simulação hemodinâmica de modelos de árvores arteriais 1D. Ela contempla o ambiente de interface gráfica (\textit{IGU}) e de console (\textit{InGU}). Simulações hemodinâmicas e análise de desempenho foram realizadas para verificar o potencial da ferramenta.

Em relação a hemodinâmica, os resultados apresentados na Seção~\ref{sec:simulacoes} estão de acordo com aqueles obtidos por Duan e Zamir\textcolor{blue}{ \cite{Duan1992} } considerando a propagação de uma onda harmônica simples nos três cenários abordados nas simulações. 

As curvas mostram a ocorrência de picos de pressão ao longo de um modelo árvore arterial. Além disso, como a viscosidade sanguínea, frequência e viscoelasticidade da parede do vaso afetam a onda de pressão.

No tocante ao desempenho computacional, foi calculado ganho de desempenho (\textit{speed up}) tanto levando em conta o número de threads quanto à carga de trabalho. Percebeu-se que a divisão das tarefas em \textit{threads} aumenta o desempenho mas como este aumento também significa um aumento de comunicação entre as \textit{threads}, limitando o ganho de desempenho. Enquanto a divisão de tarefas em cargas de trabalho permite um ganho considerável, mas requer uma quantidade grande de memória disponível.

A ferramenta construída está disponibilizada no repositório de código aberto Bitbucket:

\begin{itemize}
	\item \textbf{Link para Repositório}: \href{http://bit.ly/2KwZ4np}{http://bit.ly/2KwZ4np} coloque também depois do link (acesso em 25/08/2022)
\end{itemize}

As informações necessárias para compilar a ferramenta computacional e seus ambientes estão inclusas no Anexo A e na página inicial do repositório.

%\textcolor{red}{IGOR: escreva um parágrafo destacando propriedades/características e potencialidades da sua ferramenta computacional. Talvez você poderá recuperar ou utilizar algo do parágrafo abaixo.} \textcolor{green}{OK}

%\textcolor{blue}{A ferramenta simula e analisa modelos de árvores arteriais, podendo ser compilada nos Sistemas Operacionais: Windows e Ubuntu(Unix). O modelo de classes permite a fácil adição e configuração do ambiente para executar este experimento sobre diferentes cenários ou até mesmo outros experimentos. Os modelos podem ter sua análise e gráficos gerados automaticamente através dos comandos da ferramenta computacional. O código possibilita o processamento concorrente de diversos modelos em um mesmo ambiente. Cada uma destas contribuições resultou numa ferramenta robusta,  ferramenta esta que além de analisar corretamente as variações de fluxo e pressão através de um modelo de árvore arterial, possibilita que as simulações sejam facilmente ajustadas e analisadas, com ou sem interface gráfica.}

%\textcolor{red}{IGOR: o parágrafo acima descreve a ferramenta descrita no Capítulo 3? Este parágrafo representa a descrição da ferramenta.} \textcolor{green}{OK}

Como trabalhos futuros, destacam-se:
\begin{itemize}
	\item Analisar hemodinamicamente modelos de árvores gerados no contexto do método CCO (\emph{Constrained Constructive Optimization}) \cite{Karch1999,Queiroz2013,Queiroz2015,Brito2017};
	\item Investigar a influência da escolha parâmetros na resposta do modelo matemático de Duan e Zamir, tais como: módulo de Young e espessura do vaso;
	\item Analisar o resultado hemodinâmico utilizando um outro esquema iterativo;
	\item Calibrar estrutura de \textit{threads} para suportar experimentos maiores e poussir um \textit{overhead} de comunicação menor;
	%\item \textcolor{red} {IGOR: descreva algo que poderia ser interessante agregar na ferramenta, pode ser mais de uma coisa (coloque em item)} \textcolor{green}{OK}.
\end{itemize}