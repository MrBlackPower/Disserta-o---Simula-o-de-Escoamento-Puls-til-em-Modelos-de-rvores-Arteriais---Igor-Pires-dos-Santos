\chapter{CONCLUSÕES E TRABALHOS FUTUROS}\label{sec:conclusoes}

%\textcolor{red}{IGOR: já fiz uma primeira revisão deste capítulo e indico em vermelho o que deve fazer para melhorar este capítulo. Tudo que acrescentar ou mudar coloque em azul.} \textcolor{green}{OK}
%\textcolor{blue}{XXXXXXXXXXXXX}.

Em relação a hemodinâmica, os resultados obtidos neste trabalho estão de acordo com aqueles obtidos por Duan e Zamir\textcolor{blue}{ \cite{Duan1992} } considerando a propagação de uma onda harmônica simples nos três cenários abordados nas simulações. 

Em destaque, visando contribuir na investigação do método desenvolvido por por Duan e Zamir, curvas de impedância de entrada do modelo de árvore canina aqui considerada foram apresentadas. Estas curvas apresentam comportamento que pode ser observado em dados experimentais.

Este trabalho resultou no desenvolvimento de uma nova ferramenta computacional descrita no Capítulo~\ref{sec:ferramenta_computacional}, que permite a simulação de escoamento sanguíneo pulsátil em modelos de árvores arteriais no contexto de Duan e Zamir. A ferramenta computacional está disponibilizada no repositório de código aberto Bitbucket:

\begin{itemize}
	\item \textbf{Link para Repositório}: \href{http://bit.ly/2KwZ4np}{http://bit.ly/2KwZ4np}
\end{itemize}

As informações necessárias para compilar a ferramenta computacional e seus ambientes esta inclusa no Anexo e na página inicial do repositório.

%\textcolor{red}{IGOR: escreva um parágrafo destacando propriedades/características e potencialidades da sua ferramenta computacional. Talvez você poderá recuperar ou utilizar algo do parágrafo abaixo.} \textcolor{green}{OK}

%\textcolor{blue}{A ferramenta simula e analisa modelos de árvores arteriais, podendo ser compilada nos Sistemas Operacionais: Windows e Ubuntu(Unix). O modelo de classes permite a fácil adição e configuração do ambiente para executar este experimento sobre diferentes cenários ou até mesmo outros experimentos. Os modelos podem ter sua análise e gráficos gerados automaticamente através dos comandos da ferramenta computacional. O código possibilita o processamento concorrente de diversos modelos em um mesmo ambiente. Cada uma destas contribuições resultou numa ferramenta robusta,  ferramenta esta que além de analisar corretamente as variações de fluxo e pressão através de um modelo de árvore arterial, possibilita que as simulações sejam facilmente ajustadas e analisadas, com ou sem interface gráfica.}

%\textcolor{red}{IGOR: o parágrafo acima descreve a ferramenta descrita no Capítulo 3? Este parágrafo representa a descrição da ferramenta.} \textcolor{green}{OK}

Como trabalhos futuros, destacam-se:
\begin{itemize}
	\item Analisar hemodinamicamente modelos de árvores gerados no contexto do método CCO (\emph{Constrained Constructive Optimization}) \cite{Karch1999,Queiroz2013,Queiroz2015,Brito2017};
	\item Investigar a influência da escolha parâmetros na resposta do método, tais como: módulo de Young e espessura do vaso;
	\item Analisar o resultado hemodinâmico utilizando um outro esquema iterativo;
	\item Quantificação do aumento/perda de velocidade ao utilizar múltiplas threads;
	%\item \textcolor{red} {IGOR: descreva algo que poderia ser interessante agregar na ferramenta, pode ser mais de uma coisa (coloque em item)} \textcolor{green}{OK}.
\end{itemize}