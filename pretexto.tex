%%%%%%%%%%%%%%%%%%%%%%%%%%%%%%%%%%%%%%%%%%%%%%%%%%%%%%%%%%%5
%
%                  INFORMAÇÕES PRÉ-TEXTUAIS
%
%----------------- Título e Dados do Autor -----------------
\titulo{Uma Ferramenta Computacional para Simulação de Escoamento Pulsátil em Modelos de Árvores Arteriais 1D}
%\titulo{Desenvolvimento e implementação de limitadores de fluxo e suas aplicações em escoamento de fluidos incompressíveis}
% \subtitulo{Subt\'itulo - opcional} % opcional
\autor{Igor Pires dos Santos} \nome{Igor} \ultimonome{dos Santos}
%
%---------- Informe o Curso e Grau -----
\bacharelado %Pode ser \bacharelado \licenciatura \especializacao \mestrado ou \doutorado
\curso{Mestrado em Modelagem Computacional} 
\dia{10} \mes{Setembro} \ano{2021} % data da aprovação
\cidade{Juiz de Fora}
%
%----------Informações sobre a Instituição -----------------
\instituicao{Universidade Federal de Juiz de Fora} \sigla{UFJF}
\unidadeacademica{Programa de Pós-Graduação em Modelagem Computacional}
\departamento{Departamento de Ciência da Computação}
%
%------Nomes do Orientador, 1o. Examinador e 2o. Examinador-
\orientador{Rafael Alves Bonfim de Queiroz}
\ttorientador{Prof. Dr}
%
\coorientador{Ruy Freitas Reis} % opcional
\ttcoorientador{Prof. Dr} % se digitado \coorientador
%
\examinadorum{Nome Examinador 1}
\ttexaminadorum{Titulação Examinador 1}
%
\examinadordois{Nome Examinador 2}
\ttexaminadordois{Titulação Examinador 2}
%
\examinadortres{Nome Examinador 3}
\ttexaminadortres{Titulação Examinador 3}
%
\examinadorquatro{Nome Examinador 4}
\ttexaminadorquatro{Titulação Examinador 4}
%
%-------- Informações obtidas na Biblioteca ----------------
%
%\CDU{536.21} \areas{1.Análise Matemática  2. Topologia.}
%\npaginas{xx}  % total de páginas do trabalho
%
%%%%%%%%%%%%%%%%%%%%%%%%%%%%%%%%%%%%%%%%%%%%%%%%%%%%%%%%
%    DADOS DE FORMATACAO DA MONOGRAFIA
%
% A instrução abaixo insere a logo do curso e da instituicao na capa da monografia. Basta comentar caso não queira os logos. Para alterar o logo da instituicao e curso, basta alterar os arquivos logoInstituicao.png e logoCurso.jpg. Caso deseje alterar os arquivos, os substituta por imagens do mesmo tamanho!
% \inserirlogo  

% \metodologia

% \retirarPaginaInicio
%
%
%
%          FIM DAS INFORMAÇÕES PRÉ-TEXTUAIS
%
%%%%%%%%%%%%%%%%%%%%%%%%%%%%%%%%%%%%%%%%%%%%%%%%%%%%%%%%%

\maketitle


%% -----------------------------------------------------------------------------

%% Obs.: Alguns acentos foram omitidos.

%\titulo{Uma Ferramenta Computacional para Simulação de Escoamento Pulsátil em Modelos de Árvores Arteriais 1D} %% Colocar, dentro de chaves {}, o t\'itulo do trabalho. Retirar % do inicio da linha seguinte se tiver subtitulo
%%\subtitulo{subt\'itulo}  %% Retirar % do in\'icio desta linha se tiver subt\'itulo 
%\autor{Igor Pires dos Santos} %%Colocar, dentro de chaves {}, o nome completo do autor
%\autorR{Pires dos Santos, Igor} %%Colocar o sobrenome do autor, separado por v\'rgula, antes do restante do nome do autor. Ex.: Santos, Maria dos
%\local{Juiz de Fora} %%Governador Valadares
%\data{2021} %%Colocar o ano da entrega. Por exemplo, 2019
%\orientador[Orientador:]{Rafael Alves Bonfim de Queiroz} %%Se precisar, troque [Orientador:] por [Orientadora:]
%\coorientador[Coorientador:]{Ruy Freitas Reis} %% Colocar ``%'' no in\'icio desta linha se n\~ao tiver coorientador. Se precisar, troque por [Cooorientadora:]. 
%\orientadorTitulo{Titula\c{c}\~ao} %%Colocar, dentro de chaves {}, a titula\c{c}\~ao do(a) orientador(a). Por exemplo, Prof. Dr.
%\coorientadorTitulo{Titula\c{c}\~ao} %%Colocar, dentro de chaves {}, a titula\c{c}\~ao do(a) cooorientador(a). 
%\instituicao{Universidade Federal de Juiz de Fora}
%\faculdade{Programa de Pós-Graduação em Modelagem Computacional} %%Colocar, dentro de chaves {}, o nome da faculdade ou do instituto.
%%\programa{Modelagem Computacional} %%Colocar, dentro de chaves {}, o nome do curso. Por exemplo: Programa de P\'os\mbox{-Gra}dua\c{c}\~ao em Matem\'atica
%\objeto{Disserta\c{c}\~ao (Mestrado)}  %%Tese (Doutorado)  %%%Trabalho de Conclus\~ao de Curso (gradua\c{c}\~ao)
%\natureza{Disserta\c{c}\~ao  %%Tese 
%	apresentada ao \insereprograma ~da   %% %%%Trabalho de conclus\~ao de curso apresentado \'a \inserefaculdade da %%%%SUBSTITUIR \'a POR ao SE FOR INSTITUTO    
%	Universidade Federal de Juiz de Fora como requisito parcial \`a obten\c{c}\~ao do 
%	t\'itulo de Mestre em  %%Doutor em    %%%grau de bacharel em 
%	Modelagem Computacional. %%Trocar Matem\'atica por outro, se precisar.
%	%\'Area de concentra\c{c}\~ao: %%PREENCHER   %%%N\~ao usar esta linha se for trabalho de conclus\~ao de curso da gradua\c{c}\~ao
%}
%
%%% Abaixo, prencher com os dados da parte final da ficha catalografica
%\finalcatalog{1. Árvores arteriais. 2. Escoamento pulsátil. 3. Hemodinâmica Computacional. I. Alves Bonfim de Queiroz, Rafael, orient. II. Dr..} %% Aqui fica 
%% escrito a palavra ``T\'itulo'' mesmo, nao o do trabalho. Se tiver coorientador, os dados ficam depois dos dados 
%%% do orientador (II. Sobrenome, Nome do coorientador, coorient.) e antes de ``II. T\'itulo'', o qual passa a ``III. T\'itulo''.
%

%%Use o comando abaixo (retirando % de %\sistautordata) apenas se for usar o sistema autor-data (n\~ao o num\'erico) para refer\^encias.
%\sistautordata   





